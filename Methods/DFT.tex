\section{Density Functional Theory}
\label{Chap:Mech:DFT}

In order to investigate interactions between hydrogen atoms and free surfaces of metals or oxides, we need to study their electronic structures. On the other hand, an accurate diffusion barrier energy also relies on a accurate reaction pathway. These can be achieved by first-principles methods. The calculation of interactions between the positively charged nuclei and negatively charged electrons is considered to be a many-body problem combining kinetic energy, interactions between nuclei and electrons, and electron-electron interactions. In principle, a quantum mechanical solution to this many-body problem with $N$ electrons in an external potential $V_{ext}(r)$ generated by the nuclei can be obtained by solving the Schr\"{o}vdinger’s equation. Based on the Born-Oppenheimer approximation, the motion of nuclei and electrons can be considered separately. Besides, $V_{ext}(r)$ can be obtained by calculating the Coulomb interactions between the atomic nuclei and electrons. The time-independent, non-relativistic Schr\"{o}dinger's equation for an N-electrons system can be written as,
\begin{subequations}
\begin{align}
\hat{H}\Psi & = E\Psi \label{Chap:Meth:DFT:eq:shdr1} \\
(-\sum_i^N \frac{\hbar^2}{2m}\nabla_i^2 + \sum_i^N V_{ext}(r) + \frac{1}{2}\sum_{i \neq j}^N\frac{1}{|r_i - r_j|}) \Psi & = E \Psi \label{Chap:Meth:DFT:eq:shdr2}
\end{align}
\end{subequations}
where $E$ denotes the system energy, $\Psi$ is the many-body wave function of electrons in the system, $\hat{H}$ is the Hamiltonian operator, $N$ is the total number of electrons in the system, $m$ is the mass of a single electron, $r_i$ denotes the position of electron $i$.


The many-body system will result in many variables in the wave function ($3N$ degrees of freedom), hence making it very difficult to solve. Due to the complexity and costly computation of directly solving Equation \ref{Chap:Meth:DFT:eq:shdr2}, \acf{DFT} provides a simpler method to transfer the many-body electron problem to a single-body problem based on the density functional of electrons. \cite{wilson1984electron, kohn1965self} First, Hohenberg and Kohn proved that for any system of interacting electrons in an external potential $V_{ext}(r)$, the potential $V_{ext}(r)$ is determined uniquely by the ground state electron density $\rho_0(r)$, which means the Hamiltonian in Equation \ref{Chap:Meth:DFT:eq:shdr1} is solely determined by the ground state electron density $\rho_0(r)$. Second, the ground state energy of the system may be obtained variationally, such that the electron density $\rho(r)$ that can minimizes the total energy is the ground state density $\rho_0(r)$ exactly. In the Kohn-Sham system, the electron density $\rho(r)$ of $N$ electrons \cite{kohn1965self} is expressed as,
\begin{subequations}
\begin{align}
\rho(r) & = \sum_i^N {|\phi_i(r)|}^2 \label{Chap:Meth:DFT:eq:ks_rho}
\end{align}
\end{subequations}
where $\phi_i(r)$ is the Kohn-Sham orbital which is solved by using the Kohn-Sham Schr\"{o}vdingerr-like equation:
\begin{subequations}
\begin{align}
(H_{KS} - \epsilon_i) \Psi_i(r) & = 0 \label{Chap:Meth:DFT:eq:ks_1}
\end{align}
\end{subequations}
where $\Psi_i(r)$ is the eigenvector and $\epsilon_i$ is the corresponding eigenvalue or orbital energy for a non-interacting electron. $H_{KS}$ is the effective Kohn-Sham Hamiltonian defined as,
\begin{subequations}
\begin{align}
  H_{KS} & = -\frac{\hbar^2}{2m}\nabla_i^2 + V_{ext}(r) + V_{Hartree}(r) + V_{xc}(r) \label{Chap:Meth:DFT:eq:ks_H} \\ 
  \hfill
  V_{Hartree}(r) & = e^2\int \frac{\rho(r)}{|r - r'|} d^3r \label{Chap:Meth:DFT:eq:ks_Hartree}\\ 
  \hfill
  V_{xc}(r) & = \frac{\partial E_{xc}(\rho(r))}{\partial \rho(r)} \label{Chap:Meth:DFT:eq:ks_xc} 
\end{align}
\end{subequations}
In Equation \ref{Chap:Meth:DFT:eq:ks_H}, $V_{ext}(r)$ is still the potential accounting for the electron-nuclei interaction as Equation \ref{Chap:Meth:DFT:eq:shdr2}, $V_{Hartree}(r)$ is the Coulomb electron-electron interaction as defined in Equation \ref{Chap:Meth:DFT:eq:ks_Hartree}, $V_{xc}(r)$ is the exchange-correlation potential as obtained by Equation \ref{Chap:Meth:DFT:eq:ks_xc}, which accounts for the nonphysical self-interaction error, alongside with other effects. This term is unknown in the Kohn-Sham equation which is the reason why it is the derivative to its energy expression. However, approximation exists and depends only on the value of $\rho(r)$ at a coordinate in space where the functional is evaluated. This method is also known as the \acf{LDA}. \cite{perdew1981self} DFT results from \ac{LDA} approximations usually yield a good estimation of atomic geometries of studied system, but may overestimate the binding energies between different species. A better solution comes from \acf{GGA}. \cite{perdew1986density} The \ac{GGA} method is local but it also incorporates the effects of inhomogeneity by including the gradient of electron density at the same coordinate. \cite{ceperley1980ground} And \ac{GGA} method usually gives good estimations of the ground state energies, so it is widely used in surface calculations and transition states calculations. There are many different implementations of GGA have been developed: for examples, 1) PW91-GGA from Perdew and Wang (PW91-GGA) \cite{perdew1992accurate}, 2) PBE-GGA from Perdew, Burke and Ernzerhof (PBE) \cite{perdew1996generalized}, and 3) revised PBE-GGA for solids (PBEsol) \cite{perdew2008restoring}. Based on these approximate functionals for exchange-correlations, the Kohn-Sham Equation \ref{Chap:Meth:DFT:eq:ks_1} can be solved based on iterative methods.