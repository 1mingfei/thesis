\section{Density Functional Theory}
\label{Chap:Mech:DFT}

In order to investigate interactions between hydrogen atoms and free surfaces of metals or oxides, we need to study their electronic structures. On the other hand, an accurate diffusion barrier energy also relies on a accurate reaction pathway. These can be achieved by first-principles methods. The calculation of interactions between the positively charged nuclei and negatively charged electrons is considered to be a many-body problem combining kinetic energy, interactions between nuclei and electrons, and electron-electron interactions. In principle, a quantum mechanical solution to this many-body problem with $N$ electrons in an external potential $V_{ext}(r)$ generated by the nuclei can be obtained by solving the Schr\"{o}vdinger’s equation. Based on the Born-Oppenheimer approximation, the motion of nuclei and electrons can be considered separately. Besides, $V_{ext}(r)$ can be obtained by calculating the Coulomb interactions between the atomic nuclei and electrons. The time-independent, non-relativistic Schr\"{o}dinger's equation for an N-electrons system can be written as,
\begin{subequations}
\begin{align}
\hat{H}\Psi & = E\Psi \label{Chap:Meth:DFT:eq:shdr1} \\
(-\sum_i^N \frac{\hbar}{2m}\nabla_i^2 + \sum_i^N V_{ext}(r) + \frac{1}{2}\sum_{i \neq j}^N\frac{1}{|r_i - r_j|}) \Psi & = E \Psi \label{Chap:Meth:DFT:eq:shdr2}
\end{align}
\end{subequations}
where $E$ denotes the system energy, $\Psi$ is the many-body wave function of electrons in the system, $\hat{H}$ is the Hamiltonian operator, $N$ is the total number of electrons in the system, $m$ is the mass of a single electron, $r_i$ denotes the position of electron $i$.


The many-body system will result in many variables in the wave function (3N degrees of freedom), hence making it very difficult to solve. Due to the complexity and costly computation of directly solving Equation. \ref{Chap:Meth:DFT:eq:shdr2}, \acf{DFT} provides a simpler method to systematically transfer the many-body electron problem to a single-body problem based on the density functional of electrons. \cite{wilson1984electron, kohn1965self}