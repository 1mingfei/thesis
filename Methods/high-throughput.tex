\section{High-throughput computations}
High-throughput computations, which enable a systematic search for corrosion inhibiting elements using a robust computational engine, are a useful alternative when experimental data is unavailable. One high-throughput computational scheme that has not been extensively applied to Mg corrosion studies is to apply \ac{DFT} calculations to explore element binding on a cathode site, especially those on transition-metal second-phase particles, in Mg and related effects on the \ac{HER} \cite{zhang2019first}. Of note are two corrosion-related, high-throughput computational studies in the literature. One investigates organic corrosion inhibitors \cite{winkler2017predicting} and the other is a search for electrocatalysts that promote, rather than inhibit, the \ac{HER} \cite{greeley2006computational}. Nevertheless, high-throughput computational procedures hold significant promise in this regard with notable recent successes including prediction of new 2D materials \cite{choudhary2017high}, photoelectrocatalysts for water oxidation \cite{yan2017solar}, lithium-ion battery cathode materials \cite{tanaka2016toward,lu2017data}, precipitate structures in metal alloys \cite{saal2013materials}, and low thermal conductivity materials for thermoelectric device applications \cite{tanaka2016toward,seko2015prediction}.