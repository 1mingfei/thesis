Metallurgists have been making every effort to test and invent new desired alloys. Currently, to fulfill requirements of the quality and combination of different properties of materials, more costly and complicated processing techniques have to be used. However, for civilian-sector industrial-scale applications, cost and sustainability are also of vital importance. We need to find new strategies to reduce both the usage of precious processing facilities and the degradation of materials quality. In my dissertation, the thermodynamic driving force and key kinetic steps during relevant alloy processing procedures are studied systematically by theoretical/simulational tools at the atomistic scale. Efforts are made to improving the Ag thin-film quality during sputtering, discovering build-in corrosion-resistant mechanism for Mg alloys, and slowing down cluster nucleation in Al alloy during natural aging to avoid costly hot-working, warm stamping procedures. First, the thermodynamic driving force of H adsorption on anion-terminated (000$\overline{1}$) surfaces of pure and doped wurtzite ZnO are investigated under varying H surface coverage conditions. This understanding provides a general way to design the desired surface reconstructions of dielectric substrates before sputtering. Second, based on the chemistry and structure of ZnO (000$\overline{1}$) surface determined previously, \acf{GCMC} simulations are conducted to understand the reason why ZnO is the best substrate option for Ag thin film deposition and mechanism to achieve thinner continuous Ag films by adding ``anchor'' sites. We use first-principle calculations to search for potential good ``anchor'' site candidates. Third, thermodynamic and \acf{HER} criteria are used for high-throughput computations to search elements that can inhibit Mg alloy corrosion through second-phase transition metal particles. In other to find elements can achieve build-in corrosion resistance for Mg alloys, our first-principles computational procedure goes across the periodic table and predicts six promising p-block elements. Fourth, a \acf{KMC} method based on this \acf{NN} model is implemented to study the early transition behavior from a supersaturated solid solution to \acf{GP} zone of Al 7000 series alloys. we demonstrate that the \acf{BEP} relationship fails to provide quantitatively accurate diffusion barriers for multi-component alloys. Then a \ac{NN} model is developed to predict diffusion barriers using thousands of \ac{DFT} calculated barriers. Besides, advanced calculation approaches like \acf{LRU} cache and \acf{LSKMC} are also implemented.