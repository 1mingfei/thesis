
In my dissertation, the thermodynamic driving forces and kinetics of critical reaction steps during advanced alloy processing are studied systematically by theoretical models and simulation tools at the atomistic scale. These efforts include improving the Ag thin-film quality during sputtering, discovering a build-in corrosion-resistant mechanism for cast Mg alloys, and slowing down cluster nucleation and growth in Al solid solution alloys during natural aging to avoid costly hot stamping procedures. First, the thermodynamic driving force of H adsorption on anion-terminated (000$\overline{1}$) surfaces of pure and doped wurtzite ZnO as dielectric substrates are investigated under varying H surface coverage conditions. Understanding of these H adsorption mechanisms provides a general way to design substrate surfaces with desired binding strengths for the Ag thin-film. Second, \acf{GCMC} simulations are conducted to simulate the deposition "kinetics" of Ag thin film on substrates, which can be constructed based on the structures and properties of H-adsorbed ZnO (000$\overline{1}$) surfaces. The results demonstrate the reason why ZnO is the most suitable substrate for Ag thin film deposition and the mechanism to achieve thinner continuous Ag films by adding "anchor'' sites on the substrate surface. We use first-principles calculations to search for potential dopant elements as good "anchor'' sites on ZnO substrates and other dopants to stabilize the Ag grain boundaries to improve the polycrystalline Ag thin-film during heat treatment. Third, the \acf{HER} as the cathodic reaction on surfaces of the second-phase transition-metal (Fe) particles can speed up the corrosion of cast Mg metals and alloy. Thus, thermodynamic criteria to slow down the HER are used for high-throughput first-principles computations to search alloying elements that can reduce HER rate to achieve build-in corrosion resistance for cast Mg alloys. Our first-principles search goes across the periodic table and discovers six p-block elements that can increase the corrosion resistance for Mg, consistent with the available experimental results. Fourth, \acf{kMC} simulations are performed to study the early transition behavior from a supersaturated solid solution to \acf{GP} zone of Al 7000 series alloys at room temperature (so-called natural aging), which is critical for their thermal-mechanical processing in automobile manufacturing. Our \ac{kMC} method includes a \acf{NN} model trained by thousands of \ac{DFT} calculations to accurately predict vacancy migration barriers in Al-Mg-Zn-based alloys. Besides, advanced modeling approaches like \acf{LRU} cache and \acf{LSKMC} are also implemented to speed up the \ac{kMC} simulations in order to directly study the natural aging of Al alloys in the realistic time scales.
