\section{Introduction}

H adsorption on solid surfaces is critical to determine the electronic \cite{pearton2010recent,friend1987electronic}, optical \cite{lee2003electrical,major1986effect}, catalytic \cite{xie2011control,haruta1989gold,levy1973platinum} and many other material applications based on surface physical and chemical properties. For example, the catalytic activities of noble metal catalysts depend on the adsorption strength of critical reaction intermediates on noble metal surfaces at steady states\cite{qi2012adsorbate}. The surface electronic structures of oxides, such as ZnO, can change significantly with a different surface coverage of H atoms, and there are still debates on whether the origin of the n-type ZnO conductivity results from adsorbed H on ZnO surfaces\cite{janotti2009fundamentals}. In addition, semiconductor oxides are used as the substrate materials for the metallic thin films, whose adhesion strengths on these substrates can be reduced significantly because of adsorbed H on substrate surfaces, resulting in thin-film dewetting and the formation of the undesired discontinued islands\cite{lin2007density,duriau2006growth}.

There have been many theoretical studies based on first-principles calculations to investigate H adsorption on both metal and oxide surfaces.  The adsorption strengths of H and other adsorbates on surfaces depend not only on the interaction between the surface and a single adsorbed atom/molecule but also on the lateral interactions between adsorbates. Usually, the lateral interactions are relatively strong for adsorbates with relatively large atomic sizes, such as oxygen (O), hydroxyl (OH) and carbon monoxide (CO)\cite{Miller09,qi2012adsorbate}. It is excepted that H atoms have relatively weak lateral interactions. First-principles calculations confirmed that H adsorption strength increases slightly and continuously when H surface coverage $\theta_{\textup{H}}$ increases from 0 \ac{ML} to 1 \ac{ML} on metal surfaces \cite{pallassana1999theoretical}. Thus, H coverage on metal surfaces at equilibrium conditions changes smoothly with H chemical potential in the reservoir, and the Langmuir model can be used to describe the adsorption isotherm of H atoms in many circumstances \cite{Benard01}.

On semiconductor surfaces, the coverage-dependent H adsorption strengths may show different characteristics compared to those on metal surfaces. H coverage on these semiconductor surfaces at equilibrium conditions can change discontinuously by varying H chemical potential in the reservoir. Surface phase diagrams of H adsorption are applied to describe the stable surface structure with different H coverage \cite{deWalle02GaN,wang2005hydrogen,lauritsen2011stabilization}. Clean semiconductor surfaces contain unpaired electrons in dangling chemical bonds and can be energetically unstable \cite{Harrison79, dulub2003novel,wander2001stability,hellstrom2017surface,calzolari2013dipolar}. These surfaces can be stabilized by either surface reconstructions or adsorptions under different environmental conditions \cite{Kaxiras87, meyer2004first,lauritsen2011stabilization,wahl2013stabilization,valtiner2009temperature,Jacobs16ZnO}, and the electron counting rule \cite{pashley1989electron} plays an important role. Recently, a simple electron counting model is developed to predict and study the half-Heusler surfaces of CoTiSb \cite{kawasaki2018simple}. Similarly, the electron counting rule can also be applied to H adsorptions on semiconductor surfaces. So far many surface phase diagrams were obtained case-by-case using first-principles calculations. Most of the reported unreconstructed surfaces with lowest energies \cite{meyer2004first,valtiner2009temperature,Jacobs16ZnO} still follow the electron counting rule. Based on the understanding of the electron counting rule, it is easy to predict equilibrium H coverage on a given surface construction. However, a general method to manipulate H surface coverage is still missing.

Many efforts have been made to study H adsorption on surfaces of ZnO \cite{Meyer03,meyer2004first,wang2005hydrogen,valtiner2009temperature,lauritsen2011stabilization, wahl2013stabilization, Jacobs16ZnO}, a wide-band-gap semiconductor widely used in the fields of catalysis, gas sensing, and optoelectronics \cite{Ozgur05_ZnO,Klingshirn07_ZnO}. There are also a few studies on surfaces of ZnO with dopant elements, like Al and Mg \cite{lin2009first, lahmer2015effect}. Since there are increasing applications of doped ZnO and other semiconductors depending on their surface structures and electronic properties \cite{Pan08_doped_ZnO, Georgekutty08_Ag_ZnO, lin2009first, Ling11_Sn-Doped, Buonsanti11_Al_ZnO, Kim14_doped, Hsu14_Ag_ZnO, Lin17_Ni_SnO2}, it is necessary to explore the general principles that guide H adsorption strengths on a wide range of pure and doped ZnO and similar semiconductor surfaces. Especially, it will be interesting to verify for these general cases the accuracy of the electron counting model, which was recently applied to determine the atomic and electronic structures of (001) surface of CoTiSb, a prototypical semiconducting half-Heusler compound\cite{kawasaki2018simple}.

When ZnO bulk structure is chopped into slabs along [0001] directions, two types of surfaces, O-terminated (000$\overline{1}$) surface, and Zn-terminated  (0001) surface are exposed as shown in Figure \ref{Chap:ZnO_H:fig:ZnO}. These two surfaces were found to have different stabilization principles\cite{lauritsen2011stabilization}: the Zn-terminated  (0001) surface usually has corrugated morphology and complex surface reconstructions because the transition-metal element Zn is more flexible with respect to bonding orientation \cite{dulub2003novel,woll2007chemistry}; the O-terminated (000$\bar{1}$) surface is usually flat on the top layer with different coverage and occupation sites for O and H atoms depending on their chemical potentials because O prefers the bonding configurations with certain bond angles and nearest neighbors \cite{meyer2004first,lauritsen2011stabilization}. In addition, many functional applications of ZnO surfaces are used in oxygen-rich environments, where there are few oxygen vacancies on ZnO (000$\bar{1}$) surface \cite{meyer2004first,lauritsen2011stabilization,wahl2013stabilization}. For these reasons, we use the simple O-terminated (000$\bar{1}$) wurtzite ZnO by its bulk-terminated ideal form (the top surface in Figure \ref{Chap:ZnO_H:fig:ZnO}) as the model system to investigate the effects of electronic structures and dopant elements on the coverage-dependent H adsorption strength in this study.