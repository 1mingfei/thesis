
As mentioned in Sec. \ref{sec:Hcoverage}, the above coverage-dependent H adsorption strengths $E_{\textup{ad}}^{\textup{H}}$ on O-terminated (000$\overline{1}$) ZnO surface may change due to different passivation methods on the Zn-terminated (0001) surface, where each Zn atom has a dangling bond with 0.5 unpaired electron in its bulk-terminated ideal form. The dangling bond of each surface Zn atom should be emptied to reach the passivated structure and eliminate the artificial charge transfer from (0001) to (000$\overline{1}$) surface according to the classical electron counting model\cite{pashley1989electron}. It can be achieved by generating a  $\frac{1}{4}$ ML Zn vacancy on (0001) surface layer (one Zn vacancy in the (2$\times$2) supercell) because the 0.5 unpaired electrons from each of 3 remaining Zn atoms on (0001) surface can transfer to each of 3 O atoms that are the first-nearest neighbors of the Zn vacancy to form the stable closed-shell electron configurations. In addition, the passivated (0001) surface can also be achieved by adding one atom, such as a pseudo-hydrogen atom with 1.5 electrons (H$_{1.5}$), to each Zn atom on (0001) surface to fill the dangling bond.

\begin{table}[!htbp]
\centering
\caption[Comparison of different passivation mechanisms for the coverage-dependent adsorption energy of H atom]{The coverage-dependent adsorption energy of H atom $E_{\textup{ad}}^{\textup{H}}$ in unit of eV on (2$\times$2) O-terminated (000$\bar{1}$) ZnO surface with different mechanisms to passivate the Zn-terminated (0001) surface, including a clean Zn-terminated surface, a $\frac{1}{4}$ ML Zn vacancy (V$_{\textup{Zn}}$) on the Zn-terminated surface and pseudo-hydrogen atoms with different numbers of valence charges (1.5, 1.0 and 0.5 electron, respectively) at each Zn site on the Zn-terminated surface (H$_{1.5}$, H$_{1.0}$ and H$_{0.5}$).}
\label{tab:pass}
\begin{tabular}{ccccc}
\hline
\hline
(eV/Atom)       & 0.25ML & 0.5ML & 0.75ML & 1ML  \\ \hline
V$_{\textup{Zn}}$     & -2.43  & -2.18 & 0.43   & 0.95 \\
H$_{1.5}$         & -2.45  & -2.28 & 0.42   & 0.96 \\
H$_{1.0}$         & -2.45  & -2.27 & 0.11   & 0.95 \\
H$_{0.5}$         & -2.38  & -1.82 & 0.43   & 0.98 \\ 
Clean Zn-terminated & -2.34  & -1.49 & 0.52   & 0.95 \\
\hline
\hline
\end{tabular}
\end{table}

Tab.\ref{tab:pass} lists the H adsorption energies on (2$\times$2) (000$\overline{1}$) ZnO calculated using different passivation methods: the addition of Zn vacancies or (pseudo-)hydrogen atoms with different numbers of valence charges (H$_{1.5}$, H$_{1.0}$, and H$_{0.5}$) on (0001) surface. 
$\frac{1}{4}$ ML Zn vacancy (V$_{\textup{Zn}}$) and H$_{1.5}$ generate almost the same $E_{\textup{ad}}^{\textup{H}}$ on (000$\bar{1}$) for all investigated H surface coverages $\theta_{\textup{H}}$. These results confirm that both methods can fully passivated the Zn-terminated surface and eliminate the artificial electron transfer in the supercell. Meanwhile, an obvious difference between  H$_{1.0}$ and H$_{1.5}$ cases can be observed for the adsorption of the third H atom in the (2$\times$2) supercell ($\theta_{\textup{H}}$ = $0.75$ ML), where $E_{\textup{ad}}$ of H$_{1.0}$ case is 0.11 eV, $\sim$ 0.3eV lower than the values for V$_{\textup{Zn}}$ and H$_{1.5}$ cases. This is because one H$_{1.0}$ atom cannot completely fully filled the bonding bond of each Zn atom on (0001). For the cases of Zn-terminated surfaces without pseudo-hydrogen atoms (Clean Zn-terminated in Tab. \ref{tab:pass}) and the cases of H$_{0.5}$, because there are large amounts of electron transfers from Zn-terminated to O-terminated surfaces, H adsorption energies on the O-terminated surfaces are much weaker than those of V$_{\textup{Zn}}$ and H$_{1.5}$ cases for $\theta_{\textup{H}}$ = $0.5$ and $0.75$ ML. Thus, in the paper, only the results corresponding to H$_{1.5}$ cases (fully passivated Zn-terminated surfaces) are reported.