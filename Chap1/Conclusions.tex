\section{Conclusions}

In summary, the coverage-dependent adsorption of hydrogen atoms on O-terminated (000$\overline{1}$) surface of wurtzite ZnO and other similar polar semiconductor surfaces behave differently than the counterparts on typical metal/alloy surfaces, where H adsorption strength usually decreases slightly and continuously (more positive values of $E_{\textup{ad}}^{\textup{H}}$ in Figure \ref{Chap:ZnO_H:fig:Ead} ) with increasing H surface coverage \cite{pallassana1999theoretical,qi2012adsorbate}. The adsorption strength of individual H atom on these semiconductor surfaces strongly depends on hydrogen coverage and surface electronic structures. If the surface is in the metallic state, the hydrogen adsorption strength is so strong that the adsorption-dissociation reaction of a single H$_2$ molecule on this surface is highly exothermic at zero K ($E_{\textup{ad}}^{\textup{H}} < -2.0 $ eV in Equation \ref{eq1}). If the surface is in semiconductor state, the hydrogen adsorption strength is so weak that the adsorption-dissociation reaction of a single H$_2$ molecule on this surface is endothermic at zero K ($E_{\textup{ad}}^{\textup{H}} > 0 $ in Equation \ref{eq1}).  The surface can be transformed from metallic to semiconductor state by either hydrogen adsorption on the surface or dopant elements in the bulk lattice to saturate unpaired electrons of anion elements on the top surface layer. The critical H surface coverage $\theta_{\textup{H}}^{\textup{c}}$ to induce such metal-semiconductor transition, which is also the equilibrium H coverage at many experimental conditions \cite{lin2007density,meyer2004first,lauritsen2011stabilization},  is determined by the classical electron counting model described in Equation \ref{eq3} \cite{pashley1989electron}. This model is confirmed by our investigations of H adsorption on (000$\overline{1}$) surfaces of ZnO with a series of doping elements (Na, Mg, Al, Ti, Fe, Sn, etc.). When doping elements (such as Be, Mg, Na) have equal or fewer valence electrons than Zn, the critical H coverage will remain unchanged or increase to higher coverages. When doping elements (such as Al, Ti, and V) have more valence electrons than Zn, the critical H coverage will decrease to lower coverages. When doping elements (such as Fe, Sn, and Pb) have multiple common charge states, the behavior of the critical H coverage will become sophisticated but still follows the general electron counting rule.

This simple electron counting model can be applied to determine and manipulate the equilibrium surface adsorption configurations for H and other adsorbates on different semiconductor polar surfaces with chemical dopant elements. It is different than the d-band model that is widely used to understand trends of H adsorption strengths on different transition metal surfaces \cite{HAMMER95,Kibler05}. Since the driving force for surface reconstruction is also from the tendency to eliminate unpaired electrons and dangling bonds, the model can be used to identify the possible adsorbate and dopant configurations to obtain stable semiconductor surface structures \cite{Kaxiras87,pashley1989electron, Jacobs16ZnO}. It also suggests that the investigations of further adsorptions and depositions of other materials on these semiconductor surfaces should consider the coexistence of adsorbed H atoms at stable configurations. For example,  in many optoelectronic applications, ZnO and other semiconductor oxides are used as the substrate materials for metallic thin films, whose adhesion strengths on these substrates can be reduced significantly because of adsorbed H on substrate surfaces, resulting in thin-film dewetting and the formation of the undesired discontinued islands \cite{lin2007density,duriau2006growth}. This study provides a critical step to understand the substrate surface structures and adsorption characteristics for further investigations on thin-film qualities and optoelectronic properties.