\section{Conclusions}
In this chapter, we studied how does substrate affect Ag thin film morphology by using \ac{GCMC} and empirical potentials. This method is expandable to other metal thin films as well as other substrates, like MgO. It only requires customized parameters to match certain properties of the interaction between metal layers and substrates. Most of the metal side can use existing empirical potentials and other interactions can be tuned by L-J potential using the method described in this chapter.

We found that only Ag films on the hexagonal substrate are robust to substrate lattice constant and bonding strength changes, and yields most \{111\} orientation Ag. Therefore, Ag thin film quality (improve texture and reduce internal defects) can be improved by increasing ZnO substrate quality.

To achieve more continuous Ag thin films with less Ag, some elements can be added as ``anchor'' sites to incoming Ag atoms. Pd, Sb, Se, Sn, and Te can be good candidates as ``anchor'' sites on the ZnO substrate. With trace amount (0.05\ac{ML}) of ``anchor'' sites on the substrate, more nuclei can be achieved, hence more continuous ultra-thin film.

We also tried to search doping elements that can segregate in Ag grain boundaries to stabilize grain size during heat treatments. First-principles calculation showed transition metals, e.g. W, do not segregation in Ag grain boundaries, which is inconsistent with experiments. We suspect that alloying elements can not only change the chemistry of the grain boundaries but also disrupt the atomistic structure of grain boundaries in alloys. Therefore, an evolutionary algorithm to search stable complex grain boundaries for binary alloy need to be used. Besides, the electronic mechanism also needs to be investigated.
