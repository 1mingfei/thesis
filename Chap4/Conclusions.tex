\section{Conclusions}
In this chapter, we studied how does substrate affect Ag thin film morphology by using \ac{GCMC} and empirical potentials. This method is expandable to other metal thin films as well as other substrates, like MgO.

We found that only Ag films on the hexagonal substrate are robust to substrate lattice constant and bonding strength changes, and yields most {111} orientation Ag. Therefore, Ag thin film quality (improve texture and reduce internal defects) can be improved by increasing ZnO substrate quality. 

In order to achieve more continuous Ag thin films with less Ag, some elements can be added as ``anchor'' sites to incoming Ag atoms. Pd, Sb, Se, Sn and Te can be good candidates as ``anchor'' sites on ZnO substrate. With trace amount (0.05\ac{ML}) of ``anchor'' sites on the substrate, more nuclei can be achieved, hence more continuous ultra-thin film.