\section{Conclusions}
In this chapter, we studied how does substrate affect Ag thin film morphology by using \ac{GCMC} and empirical potentials. This method is expandable to other metal thin films as well as other substrates, like MgO. It only requires customized parameters to match certain properties of the interaction between metal layers and substrates. Most of the metal side can use existing empirical potentials and other interactions can be tuned by L-J potential using the method described in this chapter.

\ac{GCMC} simulations on these hexagonal surfaces usually yield Ag thin films that are in the fcc phase and \{111\} orientated. Therefore, Ag thin film quality (improve texture and reduce internal defects) can be improved by increasing ZnO substrate quality.

To achieve more continuous Ag thin films with less Ag, some elements can be added as ``anchor'' sites to incoming Ag atoms. Pd, Sb, Se, Sn, and Te can be good candidates as ``anchor'' sites on the ZnO substrate. With trace amount (0.05\ac{ML}) of ``anchor'' sites on the substrate, more nuclei can be achieved, hence more continuous ultra-thin film.

We also search for alloying elements that can segregate in Ag grain boundaries to stabilize grain size during heat treatments. Current \ac{DFT} calculation shows that tungsten (W) does not segregate to any investigated Ag grain boundaries. This result is inconsistent with the experimental fact that W can stabilize Ag grain boundaries during the heat treatment. The possible origin of such inconsistency can be resolved if more accurate and representative grain boundary structures in Ag alloys are constructed in future studies.