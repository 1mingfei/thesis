\section{Introduction}
%Metal/oxide interfaces are playing critical roles around us, from as small as several nanometers to as large as a couple of tens of meters. For example, in the modern semiconductor industry, complementary metal oxide semiconductor devices (CMOS) makes possible the high packing densities of modern very large scale integrated (VLSI) circuits. Also, since the 17th century, ferroconcrete or ferro cement are used to construct buildings to get reinforced structures. In ferroconcrete, concrete is filled as a paste to the metal meshed frameworks, forming a fibre-matrix, while fibers convey structural stiffness and strength to materials. The matrix transfers the load between the fibers and also supports them under compression loading. Moreover, metal/oxide interfaces also appear in aviator engines, where oxides serve as a thermal barrier or a corrosion barrier to resist metal engine blades from failure expedited by high temperature. At last, architecture outer glasses are often deposited by low-emission layers, which are usually metals .

For architecture glasses, multi-layers of low-emission materials are stacked up on glass substrates in order to achieve better energy efficiency by minimizing the ultraviolet and infrared light that can pass through glass without compromising the amount of visible light that is transmitted. The function of a multi-layer deposition is to significantly improve the reflectivity of infrared light, though it will slightly reduce transmitivity of visible light. Most of the multi-layers are repeating dielectric/low-emission/dielectric sandwich structures.

Typical low emission layers can be Silver. The ability of material to radiate energy is known as emissivity. In general, highly reflective materials, such as silver or aluminum, have a low emissivity. For example, uncoated glass has an emissivity of 0.84, while silver has an emissivity under 0.06. Hence, reducing the emissivity of the window glass by coating Ag improves a window’s insulating properties. Dielectric layers are usually oxides, like Zinc Oxide (ZnO), which are necessary to form an inferential filter that grants the reflection of the visible wavelengths to be reduced and consequently increase the light transmission. Another reason is that the reflected fraction in the visible results in as neutral a color as possible, and in particular so that the reflection does not lead to purple stains which are against human preferences. Moreover, the choice of dielectric layers or systems of dielectric layers is such that neutrality in reflection is realized for the broadest range of angles of incidence to the glazing.

However, several problems have to be solved for the sake of getting better coating properties. First, Ag/ZnO interface has weak adhesion energy compared to many other metal/oxide interfaces. The strengths of adhesion between metals and oxides can largely determine the wetting behaviors and morphology of interfaces. The silver thin film deposited onto ZnO are usually below 10nm, and their morphology are decisive to glass optical properties. For example, islands forming, bad flatness and high density of pinholes are not satisfactory. Besides, Ag/ZnO interfaces will yield in moisture environment. Therefore, the research of energetic, morphology and stability of Ag/ZnO interfaces are necessary. Moreover, methods to improve adhesion, to control morphology as well as to increase stability are also demanding.
 