\section{Introduction}

Low-emission glasses are used for architectures widely. It is used to achieve better energy efficiency by minimizing the ultraviolet and infrared light that can pass through while not significantly compromising the amount of visible light. Multi-layer structures, as shown in Figure \ref{Chap:Ag/ZnO:fig:1a}, are used for this purpose. Most of the multi-layer structures are repeating ``dielectric layer/low-emission layer/dielectric layer'' units.

Typical low-emission layers can be noble metals, e.g. Ag. The ability of a material to radiate energy is known as emissivity. In general, highly reflective materials, such as silver or aluminum, have a low emissivity. For example, untreated glass has an emissivity of 0.84, while silver has an emissivity of under 0.06\cite{salisbury1992emissivity}. Hence, reducing the emissivity of the window glass by coating Ag can improve the insulating property of architectural glasses. Dielectric layers are necessary to form an inferential filter that grants the reflection of the visible wavelengths to be reduced and consequently increases the light transmission. Another reason to utilize dielectric layers is that the reflected fraction in the visible results in as neutral a color as possible, and in particular so that the reflection does not lead to purple stains which are against human preferences. Moreover, the choice of dielectric layers or systems of dielectric layers is such that neutrality in reflection is realized for the broadest range of angles of incidence to the glazing. And oxides, e.g. ZnO, are usually great dielectric materials.

\newpage
\begingroup
\begin{figure}[!ht]
  \centering
  \subfigure{\includegraphics[width=0.75\linewidth]{Chap4/plots/Picture1a.pdf}}\label{Chap:Ag/ZnO:fig:1a}
\caption[Illustration of multi-layer structures of architecture glass.]{Illustration of multi-layer structures of architecture glass.}
  \label{Chap:Ag/ZnO:fig1}
\end{figure}
\endgroup

However, Ag/ZnO interface has weaker adhesion energy compared to many other metal/oxide interfaces. The strengths of adhesion between metals and oxides can largely affect the wetting behaviors and morphology of interfaces. On the other hand, defects, like islands forming, high surface roughness and high density of pinholes, will all reduce the glass quality, as shown in Liu's paper \cite{liu2013lithography}. For industrial practices, Ag thin film deposited on ZnO layers is usually above 10 nm thickness so as to guarantee continuous thin film. Therefore, to achieve continuous thin film with thinner layers can improve the cost efficiency by reducing the amount of Ag needed.

In this chapter, attention was paid to mainly two parts: 1) the effect of substrate on the Ag thin film quality; 2) the effect of alloy segregation at Ag grain boundary on the Ag thin film stability. \ac{GCMC} methods will be used to study the morphology of Ag thin film on different substrates with various substrate structures, lattice constant and bonding strength to Ag atoms. The goal of this study is to simulate the morphology after annealing, hence detailed kinetics, like bombardment effects, can be ignored. Therefore, \ac{GCMC} simulation will be a suitable tool, because it is based on the thermodynamic driving force. We also search for alloying elements that can segregate at Ag grain boundaries to stabilize grain size during heat treatments. Current \ac{DFT} calculation shows that tungsten (W) does not segregate at any investigated Ag grain boundaries. This result is inconsistent with the experimental fact that W can stabilize Ag grain boundaries during the heat treatment. The possible origin of such inconsistency can be resolved if more accurate and representative grain boundary structures in Ag alloys are constructed in future studies.