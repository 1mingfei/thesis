\newpage
\section{Outline}


In this work, efforts were made to solve alloy processing problems as mentioned above. The target is to combine adding a trace amount of element and affordable processing techniques to make materials with relatively high-quality for civilian usage. Attentions are paid to achieve ultra-thin Ag films under regular vacuum conditions, to increase Mg alloy corrosion resistance, and to increase formability for high strength Al alloys. To achieve these targets, the trace amount of element added to the alloy will either change the thermodynamic driving force or kinetics during the processing procedures. Because the amount of element added is very small, they will not affect alloys' original properties significantly. The rest of this dissertation is organized as follow:

In Chapter. \ref{chap:Methods}, the computational methods, and theories in the present dissertation are discussed in detail, including 1) the background and details for first-principles calculations, 2) the \acf{GCMC} ensemble. 3) the \acf{MC} simulation and the approach to combine this method with thermodynamics and kinetics. 4) deep learning and neural network methods.

In Chapter. \ref{chap:ZnO_H}, the effects of alloying elements on the hydrogen (H) equilibrium coverage on zinc oxide (ZnO) surfaces are investigated systematically. The electronic mechanism of H equilibrium coverage on ZnO surfaces is studied. And alloying elements: Na, Mg, Al, Ti, Fe, Pb, Sn, and V are considered. The approach to determining and manipulate the equilibrium surface adsorption configurations for H is studied. Besides, the generalization of this mechanism on other different semiconductor polar surfaces with chemical dopant elements are also discussed. This study helps understand the substrate conditions for silver (Ag) thin film deposition.

Follow the previous study of H equilibrium coverage on ZnO surfaces, in Chapter. \ref{chap:Ag/ZnO}, atomistic \acf{GCMC} simulation of Ag deposition on ZnO surfaces is investigated to reach better Ag thin film quality. First, the reason why the ZnO substrate is best for Ag thin film quality is explained. Then, pseudo-atoms are used as ``anchor'' sites to help us understand what kind of alloying elements and the amount of alloying elements are needed to achieve ultra-thin Ag film under regular low vacuum conditions. Following this, first-principles high-throughput calculations are conducted to search possible ``anchor'' site choices that also meet cost-efficiency. Last but not least, efforts are made to search elements than can stabilize Ag grain boundaries during growth. Calculation results have a discrepancy with experimental one, but possible reasons are explained and new methodology is proposed.

In Chapter. \ref{chap:Mg_H}, thermodynamic and \acf{HER} criteria are used for high-throughput computations to search elements that can inhibit Mg alloy corrosion through second-phase transition metal particles. In other to find elements can achieve build-in corrosion resistance for Mg alloys, our first-principles computational procedure goes across the periodic table and predicts six promising p-block elements. Among the six elements, the top two of them (As and Ge) are in accord with recent experiments. The electronic mechanism is also understood. Moreover, the generalization ability of this high-throughput computation paradigm is also tested on other precipitates. This study provides an atomistic scale simulation approach to study the thermodynamics of H adsorbates on iron precipitate surfaces.

Chapter. \ref{chap:Al/Vac} addresses how to increase formability for high strength Al alloys. This chapter will focus on methodology development to slow down clustering during natural aging to expand the time window for mechanical forming. First, we will discuss the relationship between diffusion barriers and local environments. The difficulty of using traditional bond counting method to predict diffusion barriers for multi-component systems are explained. Second, a \acf{NN} surrogate model is introduced to predict the diffusion barrier accurately. Third, a \acf{kMC} simulation package based on \ac{NN} model with advanced acceleration methods, such as parallel schema, \acf{LRU} cache, \acf{LSKMC} is introduced. Last but not least, characterization and visualization methods are described.

All of these studies are alloy processing related. Traditionally, it is more convenient to divide a full complex reaction path into many elementary steps. In each chapter, attentions are paid to one or several key elementary steps. Chapter. \ref{chap:ZnO_H} investigates the H adsorption step; Chapter. \ref{chap:Ag/ZnO} focuses on Ag adsorption/desorption/diffusion on ZnO surfaces; Chapter. \ref{chap:Mg_H} studies H adsorption on Fe precipitate surfaces. These three chapters are investigated under the ``quasi-equilibrium'' kinetic approximations, which means we assume the thermodynamic equilibrium is achieved for each elementary step, and the kinetic barrier of each elementary step is not significantly deviation from the maximum thermodynamic driving force (kinetic barrier $E_a = 0$ if energy difference $\Delta E < 0$, $E_a \sim \Delta E$ if $\Delta E > 0$). Therefore, only the energy differences ($\Delta E$) between initial and final states will be investigated and elementary kinetics steps will be derived or simulated based on energy differences. Only Chapter. \ref{chap:Al/Vac} is related to the true kinetics and activation barrier in a single elementary step and we developed a \ac{kMC} simulation to achieve that goal.

Finally, Chapter. \ref{chap:Conc} concludes the thesis and provides some thoughts on future works.