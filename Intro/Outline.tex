\section{Outline}

In this work, efforts were made to solve problems related to alloy processing mentioned above. The target is to combine adding a trace amount of element and affordable processing techniques to make materials with relatively high-quality for large-scale commercial applications. Attentions are paid to achieve ultra-thin Ag films under regular vacuum conditions, to increase Mg alloy corrosion resistance, and to increase formability for high strength Al alloys. To achieve these targets, the trace amount of element added to the alloy will either change the thermodynamic driving force or the activation barriers related to kinetics during the processing procedures. Because the amount of alloying elements added is minimal, it is expected that such alloying would not affect the original properties of alloys significantly. The rest of this dissertation is organized as the following.

In Chapter \ref{chap:Methods}, the fundamental principles of the computational methods applied in the present dissertation are discussed in detail, including 1) the background and details for first-principles calculations, 2) the \acf{GCMC} simulations, 3) the \acf{MC} simulations, and 4) deep learning and neural network methods trained by first-principles calculation results to predict the activation barriers of vacancies in multicomponent alloys.

In Chapter \ref{chap:ZnO_H}, the effects of alloying elements on the hydrogen (H) equilibrium coverage on zinc oxide (ZnO) and other similar semiconductor surfaces are investigated systematically. This study can enhance our understanding of the semiconductor substrate conditions for silver (Ag) thin film deposition. The electronic mechanism of H equilibrium coverage on pure ZnO surfaces and those with alloying elements (Na, Mg, Al, Ti, Fe, Pb, Sn, and V) are studied. The simple bond-counting mechanism is proposed to determine and manipulate the equilibrium H adsorption configurations for different alloying surfaces. Besides, the generalization of this mechanism on other different semiconductor polar surfaces with chemical dopant elements are also discussed. This work has been published in \textit{Journal of Applied Physics} in 2018. \cite{zhang2018tuning}

Follow the previous study of H equilibrium coverage on ZnO surfaces, in Chapter \ref{chap:Ag/ZnO}, atomistic \acf{GCMC} simulations of Ag deposition on ZnO surfaces are performed to investigate the key parameters that control the quality of Ag thin films. First, the possible reasons why the ZnO substrate is appropriate for high-quality (continuous and crystalline states) Ag thin films are proposed. Then, pseudo-atoms are added on the ZnO surfaces as ``anchor'' sites in GCMC simulations to achieve ultra-thin Ag film under regular low vacuum conditions. Following this, first-principles high-throughput calculations are conducted to search suitable ``anchor'' site choices that match the requirements from the above GCMC simulations. Last but not least, efforts are made to search elements than can segregate to Ag grain boundaries (GBs) to stabilize GBs and the thin-film quality during the high-temperature heat treatment processing. So far, first-principle calculation results on GB separations have some discrepancies with the experimental observations, but possible reasons are explained, and a new methodology to resolve this inconsistency is proposed. The methodology will be published in \textit{Computational Materials Science} in 2020 as a co-authored paper. \cite{yang2020grain}

In Chapter \ref{chap:Mg_H}, because the \acf{HER} on second-phase transition metal particles is the cathodic reactions of the Galvanic corrosion in Mg alloy, thermodynamic criteria are proposed to slow down the HER rate on surfaces of these particles. Based on these criteria, we perform high-throughput first-principles calculations to search for possible alloy elements that can slow down HER to achieve build-in corrosion resistance for Mg alloys.  Our first-principles computational procedure goes across all metal and semi-metal elements in the periodic table. The results suggest that six promising p-block elements slow down the HER on the surfaces of common Fe impurities from the casting processing. The most effective two elements of them (As and Ge) are in accord with recent experiments. The electronic mechanism to reduce the HER rate on these surfaces is also discussed. Moreover, the generalization ability of this high-throughput computation paradigm is also tested on other precipitates as potential active sites for the cathodic corrosion reaction. This work has been published in \textit{Computational Materials Science} in 2019. \cite{zhang2019first}

Chapter \ref{chap:Al/Vac} focuses on the methodology development on simulating the solute clustering kinetics in multicomponent Al alloys quantitatively. To slow down solute clustering at room temperatures (so-called natural aging) after the high-temperature solid-solution treatment is crucial to expand the time window for the mechanical forming of certain high-strength multicomponent Al alloys, such as 7000 series Al-Mg-Zn alloys. Since the clustering is achieved by solute diffusion based on vacancy migration, we first discuss the relationship between vacancy migration barriers and local lattice occupation environments. It is found that the traditional bond counting method fails to predict vacancy migration barriers for the multicomponent systems. Second, a \acf{NN} surrogate model based on the first-principle training data set is introduced to predict the vacancy migration barrier accurately. Third, a \acf{kMC} simulation package based on \ac{NN} model with advanced acceleration methods, such as parallel schema, \acf{LRU} cache, \acf{LSKMC} is introduced to simulate the solute clustering kinetics in multicomponent Al alloys. Last but not least, the visualization and characterization methods of solute clusters in the kMC simulation results are investigated. The information on cluster compositions and structures lays the foundation for the studies of the clustering effects on the strengths and formability of multicomponent Al alloys in the future. The manuscript for this work is still under preparation.

All of the above studies are related to the thermodynamics and kinetics of alloy processing at the atomistic scale. These studies can be understood in an integrated way described as the following. For complex kinetics related to alloy processing, it is more convenient to divide a complicated full reaction path into many elementary reaction steps. Then the efforts are made to investigate the elementary steps, especially those critical rate-determining ones. There are two general types of investigation approaches for these elementary steps. The studies from Chapter \ref{chap:ZnO_H} to Chapter \ref{chap:Mg_H} are performed under the ``quasi-equilibrium'' kinetic approximation, which is based on the assumption that the thermodynamic equilibrium is achieved for most elementary steps except the rate-determining ones. In addition, to further simplify the first-principles calculations, the kinetic barrier of each elementary step is assumed to be not significantly deviated from the maximum thermodynamic driving force. Therefore, only the energy differences ($\Delta E$) between initial and final states of the critical elementary steps will be investigated, and the kinetics of elementary steps will be derived or simulated based on these energy differences (the kinetic barrier $E_a = 0$ if the energy driving force $\Delta E < 0$, otherwise $E_a \sim \Delta E$). On the other hand, only studies in Chapter \ref{chap:Al/Vac} are conducted based on the prediction of the accurate activation barrier and kinetics in every single elementary step. This direct approach is critical for the accurate prediction of kinetics in multicomponent alloy systems, where the shapes of the potential energy landscape are so complex that the simple assumptions on the relations between thermodynamics and kinetics may fail.  Thus, the studies in this thesis provide a comprehensive portrayal to future researchers to choose appropriate methods to investigate the alloying processing at the atomistic scale.

Chapter \ref{chap:Conc} concludes the thesis and provides some thoughts on future works.