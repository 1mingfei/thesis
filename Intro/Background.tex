\section{Background}

Humans in all times have never stop the discovery of innovative alloys. The application of alloying metals allows the metallurgist to customize the properties by combining various of properties of metals together and using the metals interactions to provide unique different alloys. Alloying basically provides the opportunity of inventing new materials where the desired properties are the target.

And the history of alloy processing is a history of mankind itself. Metallurgists have been making every efforts to test and invent new desired alloys. As the requirements of the quality and combination of different properties of materials becomes higher, more costly and complicated processing techniques have to be used. The development of materials has always been strongly guided by economic factors. Those techniques and facilities are suitable for high-technology high-profit industries, like semiconductor and aviation manufactures. However, many of those techniques are not afforable to transfer to civil industry products, like architecture glasses, automobile frames, and so on.

For example, thin film deposition manufacturing are essential for solar panels, electronic and optical devices manufactures today. Basically, thin film deposition applies a very thin film of material of $\sim$nanometers to $\sim$micrometers onto a surface to be coated, or onto a previously deposited layer to form layers. In ordered to put billions of transistors into a small cpu or build multiple layer photodiodes, advanced depostion methods, like chemical vepor deposition (CVD), physical vepor deposition (PVD) or atomic layer depositon (ALD), are used. In these methods, ultra-high vacuum status will be reached in the chamber to achive highest thin film quality. However, it is impossible to apply those techniques to large scale coating on architectural glasses for low-emissivity or anti-glarizing purpose.

Not only 2D materials, bulk materials also need some expensive ``magic'' to make materials to be easily machined. For example, structural alloys need high specific strength and long product life cycles. They also require special shape to function properly. The ability to create materials of high yield strength and yet high ductility has been very chanllenging for a long time. Aluminum alloys can be subjected to costly procedures, such as controlled hot-working, warm stamping, and coupled solutioning-quenching-stamping operations in a narrow time window, followed by artificial aging to achieve precipitate hardening. The procedures described above is widely used in aerospace industry but not suitable for automobile industry or other civil industry need high strength-weight-ratio.

In this work, effors were made to solve alloy processing problems as mentioned above. The target is to combine alloying and
