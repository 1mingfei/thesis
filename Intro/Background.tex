\section{Background}

Humans in all times have never stopped the discovery of innovative alloys. The application of alloying metals allows the metallurgist to customize the properties by combining various properties of metals together and using the metals interactions to provide unique different alloys. Alloying basically provides the opportunity of inventing new materials where the desired properties are the target.

And the history of alloy processing is a history of mankind itself. Metallurgists have been making every effort to test and invent new desired alloys. As the requirements of the quality and combination of different properties of materials become higher, more costly and complicated processing techniques have to be used. The development of materials has always been strongly guided by economic factors. Those techniques and facilities are suitable for high-technology high-profit industries, like semiconductor and aviation manufactures. However, many of those techniques are not affordable to transfer to civil industry products, like architecture glasses, automobile frames, and so on.

For example, thin-film deposition manufacturing is essential for solar panels, electronic and optical devices manufacturers today. Basically, thin-film deposition applies a very thin film of materials of $\sim$nanometers to $\sim$micrometers onto a surface to be coated, or onto a previously deposited layer to form layers. In order to put billions of transistors into a small CPU or build multiple-layer photodiodes, advanced deposition methods, like chemical vapor deposition (CVD), physical vapor deposition (PVD) or atomic layer deposition (ALD), is used. In these methods, ultra-high vacuum status will be reached in the chamber to achieve the highest thin film quality. However, it is impossible to apply those techniques to large scale coating on architectural glasses for low-emissivity or anti-glaring purposes.

Not only 2D materials, but bulk materials also need some expensive ``magic'' to make materials to be easily machined. For example, structural alloys need high specific strength and long product life cycles. They also require a special shape to function properly. The ability to create materials of high yield strength and yet high ductility has been very challenging for a long time. Aluminum alloys can be subjected to costly procedures, such as controlled hot-working, warm stamping, and coupled solutioning-quenching-stamping operations in a narrow time window, followed by artificial aging to achieve precipitate hardening. The procedures described above are widely used in the aerospace industry but not suitable for the automobile industry or other civil industries that need high strength-weight-ratio.

On the other hand, corrosion resistance is another important property for alloys. Usually, the corrosion-resistant coating is used in automobile industries. However, the ability to resist corrosion decreases during serving, especially if small scratches appear. This traditional coating protection alone increases the cost of maintenance.