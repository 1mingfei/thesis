\section{Background}

Humans in all times have never stopped the discovery of innovative metallic alloys with unique properties. These targets can only be achieved by tuning the alloying compositions and by optimizing the corresponding processing procedures. In fact, the history of alloy processing is a history of humankind itself. Metallurgists have been making every effort to invent new desired alloys with appropriate processing methods. As the requirements of the quality and combination of different properties of materials become more demanding, more costly and complicated processing techniques have to be used. The development of materials has always been strongly guided by economic factors. Those techniques and facilities are suitable for high-profit industries, like semiconductor and aviation manufactures. However, many of those techniques are not affordable to transfer to low-profit industry products, like architecture glasses, automobile frames, and so on.

For example, thin-film deposition manufacturing is essential for solar panels, electronic, and optical devices manufacturers today. Basically, a very thin layer of materials with $\sim$nanometers to $\sim$micrometers in thickness is deposited onto a surface to be coated. In order to put billions of transistors into a small CPU or build multiple-layer photodiodes, advanced deposition methods, like chemical vapor deposition (CVD), physical vapor deposition (PVD) or atomic layer deposition (ALD), is used. In these methods, ultra-high vacuum status will be reached in the chamber to achieve the highest thin film quality. However, it is impossible to apply those techniques to large scale coating on architectural glasses for low-emissivity or anti-glaring purposes. If a similar thin-film quality could be achieved during sputtering, it will largely reduce the cost.

Not only 2D materials, but bulk materials also need some expensive ``magic'' to make materials to be easily manufactured. For example, structural alloys need high specific strength and long product life cycles. They are also required to have good ductility and formability to be mechanically processed into a special shape to function properly. The ability to create materials of high yield strength and yet high ductility has been very challenging for a long time. Some high-strength aluminum alloys can be subjected to costly procedures, such as controlled hot-working, warm stamping, and coupled solutioning-quenching-stamping operations in a narrow time window, followed by artificial aging to achieve precipitate hardening. The procedures described above are widely used in the aerospace industry but not suitable for the automobile industry or other civilian-sector industries that need high strength-weight-ratio. If the costly hot-working, warm stamping could be finished at room temperature, the automobile industry can be benefited from using low weight Mg alloys, thus increasing fuel efficiency.

On the other hand, corrosion resistance is another important property for alloys. Usually, the corrosion-resistant coating is used in automobile industries. However, the ability to resist corrosion decreases during serving, especially if small scratches appear. This traditional coating protection alone increases the cost of maintenance and lowers sustainability.