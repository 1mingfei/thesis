\section{Summary}

Currently, to fulfill requirements of the quality and combination of different properties of materials become higher, more costly and complicated processing techniques have to be used. However, for civilian-sector industrial-scale applications, cost and sustainability are also of vital importance. We need to find new strategies to reduce both the usage of precious processing facilities and the degradation of materials quality. In my dissertation, the thermodynamic driving force and key kinetic steps during relevant alloy processing procedures were studied systematically by theoretical/simulational tools at the atomistic scale. In Chapter. \ref{chap:ZnO_H} and \ref{chap:Ag/ZnO}, efforts are made to improving the Ag thin-film quality during sputtering by changing substrate structures and chemistry. In Chapter. \ref{chap:Mg_H}, we focus on the Mg alloy corrosion issue, and create a build-in corrosion-resistant mechanism.
In Chapter. \ref{chap:Al/Vac}, efforts are made to slow down cluster nucleation in Al alloy during natural aging in order to avoid costly hot-working, warm stamping procedures. The main conclusions include:

In Chapter. \ref{chap:ZnO_H}, the thermodynamic driving force of H adsorption on anion-terminated (000$\overline{1}$) surfaces of pure and doped wurtzite ZnO are investigated under varying H surface coverage conditions. A $\frac{1}{2}$ \ac{ML} of adsorbed H changes the electronic structure of pure ZnO (000$\overline{1}$) surface from metallic to semiconductor state by saturating unpaired electrons of surface oxygen atoms. This closed-shell electron configuration of ZnO (000$\overline{1}$) surface significantly reduces the adsorption strengths of subsequent H atoms, making the dissociative adsorption of a hydrogen molecule endothermic. A simple electron counting model is applied to predict and tune the coverage-dependent H adsorption strengths on semiconductor surfaces. When doping elements (such as Al, Ti, and V) have more valence electrons than Zn, the critical H coverage will decrease to lower coverages. We also expand the method of tuning H equilibrium coverage to other similar polar semiconductors, such as wurtzite GaN (000$\overline{1}$), and zincblende ZnS ($\overline{1}$$\overline{1}$$\overline{1}$) surfaces. This method provides a general way to generate desired surface reconstructions of dielectric substrates before sputtering.

In Chapter. \ref{chap:Mg_H}, our computational procedure predicts that six p-block elements meet thermodynamic stability and H adsorption criteria, and they rank according to their ability to reduce H adsorption energies and the \ac{HER} rate as follows: $\text{As} > \text{Ge} > \text{Si} > \text{Ga} > \text{P} \approx \text{Al}$. Results for As, the most effective corrosion-inhibiting element, and Ge are in qualitative accord with recent experiments. While none of the 68 elements was found to enhance H adsorption, the six p-block elements reduce H adsorption via strong orbital overlap (Pauli repulsion) between their outer-shell orbitals and the s orbitals of H adsorbates. We also extend our model to other major precipitates in Mg alloy. This chapter solves the problem of applying My alloys to the civilian-sector industry by improving the sustainability of alloys.

In Chapter. \ref{chap:Ag/ZnO}, we used \ac{GCMC} simulation to understand the reason why ZnO is the best substrate option for Ag thin film deposition. The hexagonal substrate, like ZnO (000$\overline{1}$), are robust to lattice constant and bonding strength changes and yields most \{111\} orientation Ag thin films. Besides, to achieve more continuous Ag thin films with less use of Ag, elements, like Pd, Sb, Se, Sn, and Te, can be added as ``anchor'' sites to incoming Ag atoms. With 0.05\ac{ML} of ``anchor'' sites on the substrate, more nuclei can be achieved. We also search for alloying elements that can segregate in Ag grain boundaries to stabilize grain size during heat treatments. \ac{DFT} calculation shows that tungsten (W) does not segregation in Ag grain boundaries, which is inconsistent with experiments. We suspect that alloying elements can not only change the chemistry of the grain boundaries but also disrupt the atomistic structure of grain boundaries in alloys.

In Chapter. \ref{chap:Al/Vac}, we first demonstrate that the \acf{BEP} relationship fails to provide quantitatively accurate diffusion barriers for multi-component alloys. Then we developed a \ac{NN} model to predict diffusion barriers using thousands of \ac{DFT} calculated barriers. And a \ac{KMC} method based on this \ac{NN} model is used to study the early transition behavior from a supersaturated solid solution to \ac{GP} zone of Al 7000 series alloys. A local super-basin method together with \ac{LRU} cache is also implemented to accelerate \ac{KMC} simulations.


\section{Future Work}

First, in Chapter. \ref{chap:Mg_H}, we discuss the possibility of using six p-elements to ``poisson'' Fe precipitates surfaces of Mg alloys. However, the top two most efficient elements are As and Ge. Even though they will achieve the best effects, As is toxic and Ge is still relatively expensive. It will be worthwhile to explore the combination of two alternative elements that can achieve a similar effect.

Second, efforts are made, in Section. \ref{Chap:Ag/ZnO:GB}, to search for potential elements that can stabilize Ag grain boundaries. Traditionally, the theoretical approach to study the alloy segregation effects is firstly obtained relaxed grain boundary structures from pure metals and then substitute dilute atoms to calculate alloy segregation energies. One possibility of the discrepancy we observed could be that alloying elements can not only change the chemistry of the grain boundaries but also change the atomistic structure of grain boundaries in alloys. Besides, in reality, more complicated grain boundaries, like grain boundary complexions, exist. \cite{cantwell2014grain} Therefore, more complicated grain boundary structures need to be obtained by global optimization methods, e.g. evolutionary algorithm, to investigate the alloy segregation effects.

Third, in Section. \ref{Chap:Al/Vac:section:KMC}, different approaches, such as \ac{LRU} cache, \ac{LSKMC}, parallel computing, are implemented to speed up the simulation for a longer time domain. So far, the simulated time span can be achieved is still at the level between $\sim$seconds to $\sim$minutes. This is still not fast enough to investigate the entire process of the \ac{GP} zone clustering. This further speedup can be achieved by:

\begin{enumerate}

  \item One of the speed limitations of our current model lies in the complexity of our \ac{NN} architecture. If a smaller architecture with fewer weight parameters can be achieved, then during the serving stage each prediction step will be more efficient.

  \item Our program is written in C++ with pure \ac{MPI}, which means one \ac{MPI} process on each core. The optimal efficiency can be achieved by using $12*N$ cores, where $N$ is the number of vacancies in the configuration. In this parallel schema, one process(core) takes care of one possible jumping event, respectively. To speed up, we can combine \ac{MPI} with OpenMP. OpenMP is a shared memory multiprocessing library. In this hybrid schema, we can use $12*N*M$, where $M$ is the number of threads per core. Then one event can be calculated by $M$ threads.

  \item \ac{LRU} cache exhibits a great speed up by looking up existing keys of encoding swiftly. However, it only works in a serial version for now. One problem needs to be solved before multi-core \ac{LRU} cache can be implemented. The problem lies in different cores do not share the same memory. Thus each core can only store one specific event out of the 12 possible events. In the consecutive iterations, the probability of a core encounter with the same event is largely reduced. Thus the time of executing one step will be determined by the slowest one. One strategy is to transfer several most recent data across the cores once several steps. In this way, all the cores will cover all the 12 possible events of recent steps.

\end{enumerate}

Fouth, we would like to continue to investigate the intrinsic mechanisms of GP zone clusters nucleation and growth kinetics. Although the sensitivity tests in \ref{Chap:Al/Vac:pseudo} yields qualitatively results for the effect of pseudo atoms, to match such an element will still take more time and computational resources. Besides, the simulation in this thesis ignores the effect of strain introduced by clustering or alloying, and only consider vacancy diffusions of an on-lattice manner. The strain effect can be implemented by using regular on-lattice \ac{KMC} with a continuum analytical function for the strain energetic contribution. The strain offsets will depend on the local concentration of the simulation cell and the concentration of the simulation cell would be flexible and interchangeable with the universe using a grand canonical ensemble mentioned in Section. \ref{Chap:Mech:GCMC}.

Fifth, in the very latest research, Liu et al. \cite{liu2020formation} found that at room temperature, \ac{GP} II zone clusters in Al-Zn-Mg alloys form very slowly. It will take a couple of weeks for the growth of \ac{GP}. However, the hardness of Al-Zn-Mg alloys still increases very quickly during natural aging. It is then critical to figure out how the short-range orderliness of clusters will affect the hardness. And the detailed mechanism of the interactions between dislocation and small clusters is also of interest.