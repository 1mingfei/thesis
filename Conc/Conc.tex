\section{Summary}
\label{sec:Summary}
The industrial-scale applications of metallic alloys with advanced structural and functional properties require the employment of efficient, low-cost, and environmental-friendly processing methods for industrial-scale manufacturing.  However, there are many unsolved problems to achieve these alloy processing methods due to complex reactions, transformation, and deformation mechanisms in the commercially-available multicomponent alloys.  In my dissertation, the thermodynamic driving force and kinetics of key reaction steps during relevant alloy processing procedures are studied systematically by theoretical/simulational tools at the atomistic scale. In Chapter \ref{chap:ZnO_H} and \ref{chap:Ag/ZnO}, multiscale simulation efforts are made to improving the Ag thin-film quality during sputtering by changing substrate structures and chemistry. In Chapter \ref{chap:Mg_H}, we focus on the Mg alloy corrosion due to impurities from casting processing and create a build-in corrosion-resistant mechanism by high-throughput first-principles calculations. In Chapter \ref{chap:Al/Vac}, multiscale simulations are performed to search the methods to slow down solute cluster nucleation and growth in Al alloy during natural aging in order to avoid costly hot stamping procedures for 7000 series Al alloys. The main conclusions are summarized as the following.

In Chapter \ref{chap:ZnO_H}, the thermodynamic driving force of H adsorption on anion-terminated (000$\overline{1}$) surfaces of pure and doped wurtzite ZnO, which is the dielectric substrate for Ag thin film deposition by sputtering, are investigated under varying H surface coverage conditions. A $\frac{1}{2}$ \ac{ML} of adsorbed H changes the electronic structure of pure ZnO (000$\overline{1}$) surface from metallic to semiconductor state by saturating unpaired electrons of surface oxygen atoms. This closed-shell electron configuration of ZnO (000$\overline{1}$) surface significantly reduces the adsorption strengths of subsequent H atoms, making the dissociative adsorption of a hydrogen molecule endothermic. A simple electron counting model is applied to predict and tune the coverage-dependent H adsorption strengths on doped ZnO surfaces. The equilibrium H coverage, above which the dissociative adsorption of a hydrogen molecule is endothermic, will decrease when doping elements (such as Al, Ti, and V) have more valence electrons than Zn. We also expand the method of tuning H equilibrium coverage to other similar polar semiconductors, such as wurtzite GaN (000$\overline{1}$), and zincblende ZnS ($\overline{1}$$\overline{1}$$\overline{1}$) surfaces. This method provides a general way to generate desired surface configurations of dielectric substrates before sputtering.

In Chapter \ref{chap:Ag/ZnO}, we utilize \acf{GCMC} simulation to understand the possible reason why ZnO is the appropriate substrate for Ag thin film deposition. The quality of Ag thin film on the hexagonal surfaces of the substrate, like ZnO (000$\overline{1}$), is robust to the variations of the substrate lattice constant and the Ag adsorption strength. \ac{GCMC} simulations on these hexagonal surfaces usually yield Ag thin films that are in the fcc phase and \{111\} orientated. Besides, to achieve more continuous Ag thin films with less use of Ag, the doping elements, like Pd, Sb, Se, Sn, and Te, can be added as ``anchor'' sites on ZnO surfaces to bind incoming Ag atoms. With 0.05\ac{ML} of ``anchor'' sites on the substrate, sufficient nuclei can be generated to achieve continuous ultra-thin films. We also search for alloying elements that can segregate in Ag grain boundaries to stabilize grain size during heat treatments. Current \ac{DFT} calculation shows that tungsten (W) does not segregate to any investigated Ag grain boundaries. This result is inconsistent with the experimental fact that W can stabilize Ag grain boundaries during the heat treatment. The possible origin of such inconsistency can be resolved if more accurate and representative grain boundary structures in Ag alloys are constructed in future studies.

In Chapter \ref{chap:Mg_H}, our computational procedure predicts that six p-block elements meet the thermodynamic criteria to slow down the \acf{HER} as the cathodic reaction on surfaces of Fe impurities and increase the build-in corrosion resistance of the cast Mg alloys. These six elements rank according to their ability to reduce H adsorption energies and the \ac{HER} rate as follows: $\text{As} > \text{Ge} > \text{Si} > \text{Ga} > \text{P} \approx \text{Al}$. Results for As, the most effective corrosion-inhibiting element, and Ge are in qualitative accord with recent experiments. While none of the 68 investigated alloying elements was found to enhance H adsorption on Fe surfaces, the six p-block elements reduce H adsorption on Fe surface via strong orbital overlap (Pauli repulsion) between their outer-shell orbitals and the s orbitals of H adsorbates. We also extend our model to search the possible alloying elements to impede \ac{HER} on surfaces of other possible transition-metal impurities and precipitates in Mg alloys. 

In Chapter \ref{chap:Al/Vac}, Ag-Mg-Zn ternary alloy is used as a model system to simulate the solute clustering kinetics in Al 7000 series alloys. We first demonstrate that the \acf{BEP} relationship, which suggests a simple linear relation between the activation barrier and the reaction energy of one elementary reaction step, fails to provide quantitatively accurate migration barriers of vacancies in these multi-component Al alloys. Then we develop a \ac{NN} model to predict vacancy migration barriers using the training data set of thousands of \ac{DFT} calculated barriers for different alloy configurations. A \ac{kMC} method based on this \ac{NN} model is used to study the early transition behavior from a supersaturated solid solution to solute clusters and \acf{GP} zones in Al-Mg-Zn alloys. A local super-basin method  in Section \ref{Chap:Al/Vac:sec:LSKMC}, together with \ac{LRU} cache in Section \ref{Chap:Al/Vac:sec:LRU}, is also implemented to accelerate \ac{kMC} simulations. We also propose a pseudo-atoms approach to efficiently search the alloying strategy to slow down the solute clustering and the corresponding natural aging effects in Al 7000 series alloys. In this approach, a small number of pseudo-atoms with artificially designed ability to change vacancy migration barriers are added into the Al matrix, and the \ac{kMC} simulations are performed to check their effects to clustering kinetics (so-called ``sensitivity test''). We also develop the quantitative analysis methods to describe the chemical and structural properties of clusters. At last, we propose a machine learning strategy based on the structural and chemical information of clusters and precipitates from \ac{kMC} simulations to predict the cluster strengthening and natural aging effects in future studies.

\section{Future Work}

First, in Chapter \ref{chap:Mg_H}, we discuss the possibility of using six p-elements to ``poison'' surfaces of Fe precipitates in cast Mg alloys to slow down the Galvanic corrosion. However, for the top two most efficient elements, As is toxic, and Ge is still relatively expensive and less effective compared with As. It will be worthwhile to explore the combination of two alternative alloying elements that can achieve a similar or even improved effect to impede the cathodic reactions and the overall corrosion rates.

Second, in Section \ref{Chap:Ag/ZnO:GB}, efforts are made to search for potential elements that can stabilize Ag grain boundaries during heat treatment. Traditionally, the theoretical approach to study the alloy segregation effects is firstly obtained relaxed grain boundary structures from pure metals and then substitute the matrix atoms by solute atoms to calculate alloy segregation energies. One possible origin of the discrepancy between DFT calculations and experiments mentioned in Section \ref{Chap:Ag/ZnO:GB} could be that alloying elements change not only the chemistry of the grain boundaries but also the atomistic structure of grain boundaries in alloys. Besides, in reality, more complicated grain boundaries, like grain boundary complexions, exist\cite{cantwell2014grain}. Therefore, more complicated grain boundary structures need to be obtained by global optimization methods, e.g., evolutionary algorithm \cite{yang2020grain}, to investigate the alloy segregation effects.

Third, in Section \ref{Chap:Al/Vac:section:KMC}, different approaches, such as \ac{LRU} cache in Section \ref{Chap:Al/Vac:sec:LRU}, \ac{LSKMC} in Section \ref{Chap:Al/Vac:sec:LSKMC}, parallel computing, are implemented to speed up the \ac{kMC} simulations for a longer time. So far, the simulated time span that can be achieved is still at the level between $\sim$seconds to $\sim$minutes. This is still not sufficient enough to simulate the entire process of the solute clustering and \ac{GP} zone formation during the natural aging. Further speedup of the \ac{kMC} simulations could be achieved by the following methods.

\begin{enumerate}

  \item One of the speed limitations of our current \ac{kMC} simulation lies in the complexity of our \ac{NN} architecture. If a smaller architecture with fewer weight parameters can be achieved, each prediction step will be more efficient during the simulations.

  \item Our program is written in C++ with pure \ac{MPI}, which means one \ac{MPI} process on each core. The optimal efficiency can be achieved by using $12*N$ cores. Here $N$ is the number of vacancies in the \ac{kMC} simulation supercell and 12 is the number of first-nearest-neighbor sites that a vacancy can jump to. In this parallel schema, one process(core) takes care of one possible jumping event, respectively. To speed up, we can combine \ac{MPI} with OpenMP. OpenMP is a shared memory multiprocessing library\cite{dagum1998openmp}. In this hybrid schema, we can use $12*N*M$, where $M$ is the number of threads per core. Then one event can be calculated by $M$ threads.

  \item \ac{LRU} cache exhibits a significant speed-up effect by looking up existing keys of encoding swiftly. However, it only works in a serial version for now. One problem needs to be solved before multi-core \ac{LRU} cache can be implemented. The problem lies in different cores do not share the same memory. Thus each core can only store one specific event out of the 12 possible events for one vacancy migration. In the consecutive iterations, the probability of a core encounter with the same event is largely reduced. Thus the time of executing one step will be determined by the slowest one. One strategy is to transfer several most recent data across the cores once several steps. In this way, all the cores will cover all the 12 possible events of recent steps.

\end{enumerate}

Fourth, the \ac{kMC} simulation in this thesis ignores the effect of strain introduced by clustering or alloying. The strain effects on solute/vacancy migration barriers and clustering kinetics still remain unknown. The strain effect can be implemented by using regular on-lattice \ac{kMC} with a continuum analytical function for the strain energetic contribution. The strain offsets could depend on the local concentration of the simulation cell, and the concentration of the simulation cell would be flexible and interchangeable with the external reservoir using a grand canonical ensemble mentioned in Section \ref{Chap:Mech:GCMC}.

Fifth, we would like to continue to investigate the intrinsic mechanisms of the nucleation and growth kinetics for solute clusters and early-stage GP zones based on the sensitivity tests using pseudo atoms in Section \ref{Chap:Al/Vac:pseudo}. Once the requirements for the optimal pseudo atoms to slow down the cluster kinetics are identified, high-throughput DFT calculations will be performed to find the corresponding real alloy elements. To finally quantify the effects of these alloy elements in strengths and formability of Al alloys during natural aging, a cluster strengthening model will be constructed based on chemical, structural, and energetic properties of the solute clusters obtained from \ac{kMC} simulations discussed in the next paragraph.

Sixth, details about how clustering in Al alloy increases strength is still unclear. For typical solid solution strengthening model, the effects of individual solute atoms on both elastic interactions with dislocation strain fields (``strain misfit effects'') and the stacking fault energies related to dislocation core behavior (``chemical effects'') were considered to predict the solute strengthening effects based on first-principles calculations\cite{yasi2010first}. For the clustering strengthening model, there was a model that only considers the elastic strain effects of clusters\cite{zhao2014cluster}. In multicomponent alloys, the elastic strain effects and chemical effects of solute clusters could couple together and become more complicated when the chemical compositions and structures of clusters become complex. First-principles calculations can be applied to obtain the parameters related to elastic strain and chemical effects of these clusters, similar to those for solid solution strengthening model\cite{yasi2010first}.  Then these parameters will be used in the classical continuum model to predict the overall cluster strengthening effects. Advanced machine learning models can be used to speed up the prediction based on first-principle data sets of the parameters of elastic strain effects and chemical effects of many clusters from \ac{kMC} simulations. Then we can build a surrogate model to predict the natural aging hardening effects under the influence of various types of solute elements directly based on \ac{kMC} simulation results.

According to previous studies, the chemical effect of clusters could be more important than the strain effects for the overall strengthing. For example, in the very latest research, Liu et al. \cite{liu2020formation} found that at room temperature, \ac{GP} II zone clusters in Al-Zn-Mg alloys form very slowly. It will take a couple of weeks for the growth of \ac{GP} zone clusters. However, the hardness of Al-Zn-Mg alloys still increases very quickly during natural aging. The authors claim that solute cluster hardening is accountable for the rapidly hardening effect of the alloy. When dislocations cut through an ordered solute cluster, only local bond changes will be observed. Additional energy is required to create an anti-phase boundary (APB). It is then critical to figure out how the short-range order of clusters affects the strengthening results.  Such studies require analyses of chemical bonds and electronic structures based on first-principles calculations. 