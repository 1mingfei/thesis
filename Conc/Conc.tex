\section{Summary}

\section{Future Work}

First, in Chapter. \ref{chap:Mg_H}, we discussed the possibility of using six p-elements to ``poisson'' Fe precipitates surfaces of Mg alloys. However, the top two most efficient elements are As and Ge. Even though they will achieve the best effects, As is toxic and Ge is still relatively expensive. It will be worthwhile to explore the combination of two alternative elements that can achieve a similar effect.

Second, efforts were made, in Section. \ref{Chap:Ag/ZnO:GB}, to search potential elements that can stabilize Ag grain boundaries. Traditionally, the theoretical approach to study the alloy segregation effects is firstly obtained a relaxed grain boundary structures from pure metals and then substitute dilute atoms to calculate alloy segregation energies. One possibility of the discrepancy we observed could be that alloying elements can not only change the chemistry of the grain boundaries but also change the atomistic structure of grain boundaries in alloys. Besides, in reality, more complicated grain boundaries, like grain boundary complexions, exist. \cite{cantwell2014grain} Therefore, more complicated grain boundary structures need to be obtained by global optimization methods, e.g. evolutionary algorithm, to investigate the alloy segregation effects.

Third, in Section. \ref{Chap:Al/Vac:section:KMC}, different approaches, such as \ac{LRU} cache, \ac{LSKMC}, parallel computing, were implemented to speed up the simulation for a longer time domain. So far, the simulated time span can be achieved is still at the level between $\sim$seconds to $\sim$minutes. This is still not fast enough to investigate the entire process of the \ac{GP} zone clustering. This further speedup can be achieved by:

\begin{enumerate}

  \item One of the speed limitations of our current model lies in the complexity of our \ac{NN} architecture. If a smaller architecture with fewer weight parameters can be achieved, then during the serving stage each prediction step will be more efficient.

  \item Our program is written in C++ with pure \ac{MPI}, which means one \ac{MPI} process on each core. The optimal efficiency can be achieved by using $12*N$ cores, where $N$ is the number of vacancies in the configuration. In this parallel schema, one process(core) takes care of one possible jumping event, respectively. To speed up, we can combine \ac{MPI} with OpenMP. OpenMP is a shared memory multiprocessing library. In this hybrid schema, we can use $12*N*M$, where $M$ is the number of threads per core. Then one event can be calculated by $M$ threads.

  \item \ac{LRU} cache exhibits a great speed up by looking up existing keys of encoding swiftly. However, it only works in a serial version for now. One problem needs to be solved before multi-core \ac{LRU} cache can be implemented. The problem lies in different cores do not share the same memory. Thus each core can only store one specific event out of the 12 possible events. In the consecutive iterations, the probability of a core encounter with the same event is largely reduced. Thus the time of executing one step will be determined by the slowest one. One strategy is to transfer several most recent data across the cores once several steps. In this way, all the cores will cover all the 12 possible events of recent steps.

\end{enumerate}

Fouth, we would like to continue to investigate the intrinsic mechanisms of GP zone clusters nucleation and growth kinetics. Although the sensitivity tests in \ref{Chap:Al/Vac:pseudo} yields qualitatively results for the effect of pseudo atoms, to match such an element will still take more time and computational resources. Besides, the simulation in this thesis ignored the effect of strain introduced by clustering or alloying, and only considered vacancy diffusions of an on-lattice manner. The strain effect can be implemented by using regular on-lattice \ac{KMC} with a continuum analytical function for the strain energetic contribution. The strain offsets will depend on the local concentration of the simulation cell and the concentration of the simulation cell would be flexible and interchangeable with the universe using a grand canonical ensemble mentioned in Section. \ref{Chap:Mech:GCMC}.

Fifth, in the very latest research, Liu et al. \cite{liu2020formation} found that at room temperature, \ac{GP} II zone clusters in Al-Zn-Mg alloys form very slowly. It will take a couple of weeks for the growth of \ac{GP}. However, the hardness of Al-Zn-Mg alloys still increases very quickly during natural aging. It is then critical to figure out how the short-range orderliness of clusters will affect the hardness. And the detailed mechanism of the interactions between dislocation and small clusters is also of interest.