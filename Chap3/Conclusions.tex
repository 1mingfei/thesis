\section{Conclusions}

A strategy for increasing the corrosion resistance of Mg with Fe impurities that has gained significant traction in the experimental literature involves finding alloying elements that slow down or even inhibit the \ac{HER} as the cathodic corrosion reaction on surfaces of Fe second-phase particles. The present study used a high-throughput DFT calculation strategy to search for such alloying elements based on three criteria. First, an element must show a thermodynamic preference for bulk BCC Fe over bulk HCP Mg; second, an element must be thermodynamically more stable on Fe (100) and Fe (110) than bulk Fe; third, the H adsorption energies on both Fe (100) and Fe (110) with an alloying element in their topmost layers should be significantly reduced or enhanced relative to the adsorption energies of a lone H atom on pure Fe (100) and (110), suggesting likely interference with the \ac{HER} rate. The major conclusions of this study are as follows:

(1) Calculations show that As as an alloying element can satisfy the above three criteria. Depending on the As surface concentration, the hydrogen adsorption energies can be significantly reduced ($>$ $\sim$10 $k_{B}T$ at the room temperature T $\sim$ 300 K), which can slow down the generation of adsorbed H atoms through water molecule dissociation (the Volmer reaction, Eq. \ref{Chap:Mg_H:eq:Volmer}) in the \ac{HER} reaction mechanisms. This is different than previous arguments where As can slow down the hydrogen recombination on Fe surfaces. The extent to which H adsorption energies are reduced depends on As concentration in the top surface layer of Fe. If one-half of the Fe atoms are substituted by As atoms in the top layer, which is thermodynamically favorable compared with As atoms in Fe bulk, then the H adsorption strength on such surface is predicted to reduce by $\sim$1 eV per H atom, even weaker than H adsorption strength on Au surfaces, which is extremely inactive for \ac{HER}.

(2) Of the 68 potential elements considered as possible corrosion inhibitors, the following six elements, rank-ordered by their ability to reduce (or destabilize) H adsorption on Fe surfaces, meet all three criteria: $\text{As} > \text{Ge} > \text{Si} > \text{Ga} > \text{P} \approx \text{Al}$. Each of the six identified elements likely inhibits the Volmer reaction \ref{Chap:Mg_H:eq:Volmer} necessary for the \ac{HER} on Fe impurities that function as cathode surfaces. Identification of As and Ge as the best of all 68 elements examined is in qualitative accord experiments These p-block elements are also found to have the potential to reduce \ac{HER} on Ni second-phase particles according to the same criteria. However, the ability of these p-block alloying elements to inhibit \ac{HER} on Cu second-phase particles is limited because these elements do not show strong preferences to be stable in Cu particles relative to those in Mg matrix.

(3) Electron density difference contours show that excess electrons accumulate in the outer-shells (e.g. s and p orbitals) of each atom of the six elements replacing a Fe atom in both Fe (100) and Fe (110). This causes an enhancement to Pauli repulsion between an adsorbed H atom at an adjacent surface site and a substitutional alloying atom when orbitals of the latter overlap with the s-orbital of an H adsorbate. A significant reduction of H adsorption energy results (or destabilization of H binding) by increasing the repulsion.