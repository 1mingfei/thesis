A significant challenge for applications of Mg alloys is their poor corrosion resistance and hence Mg alloys designs with built-in corrosion resistance are of significant interest. Corrosion can result from the coupling of anodic dissolution of Mg and cathodic reduction of water on impurities such as Iron (Fe)-rich second-phase particles. Experiments have shown that small quantities of Arsenic (As) or Germanium (Ge) can inhibit Mg corrosion, possibly slowing the \ac{HER} as the cathodic reaction on Fe surfaces. Since a broader experimental search across the periodic table for other Mg corrosion inhibiting elements is unavailable, we designed thermodynamic and \ac{HER} criteria, and used high-throughput computations to search a pool of 68 elements including As and Ge that can segregate from bulk Mg to surfaces of Fe particles and impede \ac{HER} there. Our computational procedure predicts that six p-block elements meet these criteria, and they rank according to their ability to reduce H adsorption energies and the \ac{HER} rate as follows: $\text{As} > \text{Ge} > \text{Si} > \text{Ga} > \text{P} \approx \text{Al}$. Results for As, the most effective corrosion-inhibiting element, and Ge are in qualitative accord with recent experiments. While none of the 68 elements was found to enhance H adsorption, the six p-block elements reduce H adsorption via strong orbital overlap (Pauli repulsion) between their outer-shell orbitals and the s orbitals of H adsorbates. These p-block elements are also found to have the potential to reduce \ac{HER} on surfaces of Ni second-phase particles in Mg according to the same criteria, but not on surfaces of Cu second-phase particles. 

\section{Introduction}
Magnesium alloys are potential candidates for lightweight structural components in transportation industries and other applications due to their low density and high specific strength. Substantial effort has been focused on exploring cast and wrought magnesium (Mg) alloys for application in vehicle propulsion systems and body structures, for example, as part of light-weighting strategies \cite{luo2005development, luo2006wrought,carter2011structural, jekl2015development, luo2013magnesium}. However, corrosion in aqueous and atmospheric environments is one of two significant challenges facing broader implementation of current Mg alloys in vehicles. The other challenge is poor room temperature formability of Mg sheet alloys in stamping, a consequence of the anisotropy in plasticity response associated with the various dislocation slip systems in the \ac{HCP} structure \cite{yasi2010first}. Post-forming surface treatments can be applied to mitigate Mg corrosion \cite{zheng2005corrosion}. 
In an aqueous environment, the coupling between regions of anodic dissolution of Mg and cathodic reduction of water drives galvanic corrosion leading to the removal of Mg and the formation of pits surrounding second-phase particles as cathode sites \cite{birbilis2014evidence, zeng2006review}. Magnesium has a highly negative standard electrode potential of -2.37 V relative to the \ac{SHE}, making it a very active anode (all electrode potential values are relative to the \ac{SHE} in this paper). Anodic dissolution of Mg via 
\begin{align}
Mg \rightarrow Mg^{2+} + 2e^{-}
 \label{Chap:Mg_H:eq:anodic_dissolution}
\end{align}
couples with the cathodic reaction, which is the reduction of water in an alkaline electrolyte 
\begin{align}
2H_2O + 2e^{-} \rightarrow H_{2}(g) + 2OH^{-}
 \label{Chap:Mg_H:eq:cathodic_reaction}
\end{align}
Eq. \ref{Chap:Mg_H:eq:cathodic_reaction} is the \ac{HER} with a standard electrode potential of -0.828 V relative to \ac{SHE}. The overall reaction mechanism is illustrated in Fig. 1(a). 

Corrosion on Mg proceeds without any limitation for pH < 11 since oxygen is not involved and no passivating surface layer forms \cite{liu2016controlling,ralston2012effect}. Previously, it was found that Fe, Cu, Co and Ni accelerate Mg corrosion in aqueous environments containing chlorides \cite{hanawalt1942corrosion, mcnulty1942some}. Even though Fe (an impurity often introduced during alloy processing \cite{yang2015corrosion, scharf2007iron}) has a very low solubility limit in Mg \cite{mcnulty1942some}, Fe particles in \ac{BCC} structure have been identified as cathode sites. This was demonstrated by Taub et al. [16] with experiments involving powdered Mg and Fe in chloride solutions. Cathodic reaction in Mg alloys typically occurs at submicron and micron-scale Fe (and Fe-rich) second-phase particles positioned at the metal-aqueous solution interface \cite{yang2015corrosion, eaves2012inhibition}. The detailed roles of Fe-rich particles as cathodic reaction sites can vary according to corrosion potentials (such as the \ac{NDE}) and the populations/sizes of Fe-rich particles \cite{hoche2016effect, yang2018effect}; the cathodic reaction rates should also depend on the surface configurations of Mg, which typically contain mixtures of porous Mg hydroxides and oxides (e.g. aluminum oxide). Hence, the accurate locations of Fe-rich particles relative to these mixtures can result in changes in corrosion rates \cite{taheri2012analysis, taheri2014towards}. Song and Atrens provided an overview of galvanic as well as other corrosion mechanisms of Mg with some useful insights \cite{song2003understanding}.  Esmaily et al. presented a comprehensive summary of the recent progress of studies in Mg-alloy corrosion \cite{esmaily2017fundamentals}. 
There is significant interest in designing Mg alloys that have a “built-in” corrosion inhibition mechanism \cite{eaves2012inhibition}. A possible strategy is to slow down the HER rates on cathode sites such as Fe particles in \ac{BCC} phase. The HER can be completed through either the Volmer-Heyrovsky mechanism or the Volmer-Tafel mechanism [25, 26]. The associated reactions are
\begin{subequations}
\begin{align}
&\text{Volmer reaction:    } & H_2O + e^- & \rightarrow H^* + OH^-
 \label{Chap:Mg_H:eq:Volmer}\\
&\text{Heyrovsky reaction:    } & H^* + H_2O + e^- & \rightarrow H_2(g) + OH^-
 \label{Chap:Mg_H:eq:Heyrovsky}\\
&\text{Tafer reaction:    } 
& 2H^* & \rightarrow H_2(g)
 \label{Chap:Mg_H:eq:Tafel}
\end{align}
\end{subequations}
Here * means the corresponding atom/molecule is adsorbed on cathode surface sites. The Volmer-Heyrovsky mechanism proceeds first via Eq. \ref{Chap:Mg_H:eq:Volmer} followed by Eq. \ref{Chap:Mg_H:eq:Heyrovsky}. The Volmer-Tafel mechanism involves Eq. \ref{Chap:Mg_H:eq:Volmer} twice followed by Eq. \ref{Chap:Mg_H:eq:Tafel}. The equilibrium potentials of Eqs. \ref{Chap:Mg_H:eq:Volmer}, \ref{Chap:Mg_H:eq:Heyrovsky} and \ref{Chap:Mg_H:eq:Tafel} depend on the free energies of hydrogen atoms adsorbed on cathode surface sites. 

The slowing down of the HER as the cathodic reaction requires the reduction of reaction rates via the Volmer-Heyrovsky mechanism and the Volmer-Tafel mechanism. Regarding the Volmer reaction in Eq. \ref{Chap:Mg_H:eq:Volmer}, its fast kinetics favors the strong adsorption of an H atom on a cathode surface site to increase its thermodynamic driving force since H* is its reaction product. However, the fast kinetics of the Heyrovsky and Tafel reactions in Eq. \ref{Chap:Mg_H:eq:Heyrovsky} and \ref{Chap:Mg_H:eq:Tafel}, respectively, require the weak adsorption of an H atom since H* is the reactant on the left side of each equation. Thus, the adsorption strength of H atoms on a cathode surface must be at an intermediate range to reach the maximum \ac{HER} rate: this is the Sabatier principle [27]. Figure 1(b) shows the exchange-current density of the \ac{HER} vs. H adsorption energy, $E_{ad}^H$, in alkaline electrolytes [28]: the highest values of log(i0) corresponding to the fastest \ac{HER} rate appear for the intermediate $E_{ad}^H$. Similar plots of \ac{HER} rate in acid electrolyte vs. $E_{ad}^H$ have been explored on different metals, even though the detailed reaction mechanisms in the \ac{HER} differ from those in an acid electrolyte. Therefore, to slow down the overall \ac{HER} rate on cathode sites, either the adsorption strength of H must be strongly enhanced, so that the rate of the Heyrovsky and Tafel reactions is largely decreased, or the adsorption of H must be significantly weakened thereby reducing the rate of the Volmer reaction. 