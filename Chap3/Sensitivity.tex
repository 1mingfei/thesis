\section{Sensitivity Analysis for H Adsorption Energy on Surface Alloying Coverage}

All of the above evaluations are based on the three criteria in Section 2.1 with relatively simple but well-defined Fe surface structures with alloying elements. However, the quantitative predictions of alloying effects to Mg corrosion resistance require the accurate descriptions of surface and bulk structures of Fe-rich particles. As shown in Fig. \ref{Chap:Mg_H:fig1} (b), the overall alloying effect to reduce the \ac{HER} rate depends on the hydrogen adsorption energies, which can change when the substitutional alloying concentrations vary. However, the highly negative segregation energies $E_{surf}$ seg for these six candidates calculated using one substitutional alloying atom in a (2x2) surface unit cell shown in Figs. \ref{Chap:Mg_H:fig6} and \ref{Chap:Mg_H:fig7} suggest that there are substantial thermodynamic tendencies to have higher alloying element coverage on Fe surfaces. In addition, the reductions of hydrogen adsorption energies on Fe (100) surfaces due to $\frac{1}{4}$ \ac{ML} alloy surface coverage for the six alloying candidates are not very large (0.2 $\sim$ 0.4 eV per H atom), so the values of $E_{ad}^H$  on these alloyed surfaces are still close to the region corresponding to the maximum exchange current density of HER indicated by Fig. \ref{Chap:Mg_H:fig1} (b). These $E_{ad}^H$ values cannot strongly support the argument that such alloying elements can significantly reduce HER rates on Fe surfaces and inhibit the corrosion reactions on Mg alloys. For these reasons, we investigated the effects of higher substitutional alloying concentrations in the top layer of the two Fe surfaces.


We first investigated whether it is possible to add a second substitutional alloying element in the top surface layer that already has one substitutional alloying element for a (2x2) Fe surface slab. Thus, we calculated the surface segregation energy $E_{surf}$ seg defined in Eq. \ref{Chap:Mg_H:eq:surf_seg} for the second substitutional alloying element in the top layer of (2x2) Fe (100) and (110) surface slabs, where $E_{surf}$ seg for the first alloying element already has strong negative values as shown in Figs. \ref{Chap:Mg_H:fig6} and \ref{Chap:Mg_H:fig7}. In Eq. \ref{Chap:Mg_H:eq:surf_seg}, $E_{bulk}$ is the energy of (2x2) surface slabs with one substitutional alloying element in the top surface layer and the other in the bulk far from the surface; $E_{surf}$ is the energy of a (2x2) surface slab with two substitutional alloying elements in the top surface layer. As shown in Figs. \ref{Chap:Mg_H:fig11} (a) and \ref{Chap:Mg_H:fig11} (b), there are two possible configurations of two substitutional alloying elements in a (2x2) Fe (100) surface slab. As shown in Fig. \ref{Chap:Mg_H:fig11} (c), the alloying configuration in Fig. \ref{Chap:Mg_H:fig11} (b) (“config 2”) always generates more negative $E_{surf}$ seg for the second substitutional element compared with their counterparts in Fig. \ref{Chap:Mg_H:fig11} (a) (“config 1”). Thus, it is energetically favorable for these elements to have the configuration shown in Fig. \ref{Chap:Mg_H:fig11} (b).


On these (2x2) Fe (100) surfaces, with two substitutional alloying elements in the top layer, there is at least one alloying element as the nearest neighbor for H atoms absorbed at all the hollow sites and bridge sites. Thus, the hydrogen adsorption energy should be very weak. The hydrogen adsorption energies on the hollow site in Fig. \ref{Chap:Mg_H:fig11} (b), the most favorable adsorption site on Fe (100), are indeed found to have very weak adsorption energy for the six p-block alloying element candidates. As shown in Fig. \ref{Chap:Mg_H:fig11} (d), $E_{ad}^H$ of As, Ge, Ga, P, Si, and Al are positive values, with weaker hydrogen adsorption than that on the noble metal (Au, Ag) surfaces considered in Fig. \ref{Chap:Mg_H:fig1} (b). Such significant reductions of hydrogen adsorption energies suggest that the Volmer reaction in Eq. \ref{Chap:Mg_H:eq:Tafel} is indeed slow enough to impede the \ac{HER} rate on cathode sites. Alternatively, the (2x2) Fe (100) surface with 2 substitutional Zr atoms still has H adsorption energies comparable to those on pure Fe surfaces, suggesting higher surface concentrations of Zr (possibly for other similar transition metal elements like Hf) cannot result in noticeable changes in \ac{HER} rates. Similar results apply to alloying Fe (110) surfaces.


The above DFT calculations suggest that one-half of a \ac{ML} of alloying element X (As, Ge, Ga, etc.) can effectively inhibit the HER on Fe surfaces. Usually, the concentration of Fe impurities in Mg alloys is of the order of 100 ppm. Here, we assume all Fe impurities exist as second-phase particles in a nano-cube shape and each cube has 6 {100} facets each with a length L of 10 or 100 nm. A simple analysis shows that $\sim$10 ppm or $\sim$1 ppm of alloying element X is enough to cover one-half of all Fe second-phase particle surfaces. Our calculations suggest that these alloying elements have a strong preference to segregate to Fe surfaces compared with the Mg or the Fe bulk lattice according to our DFT results, we conclude that $\sim$10 ppm level of these alloying elements will be sufficient to show a noticeable effect on Mg corrosion resistance if there are no other phases/locations that can strongly attract these alloying elements. In reality, other stable occupation sites for these alloying elements could be on different precipitate phases and grain boundaries, so a higher concentration of alloying element X may be required to significantly enhance the corrosion resistance of Mg alloys.
