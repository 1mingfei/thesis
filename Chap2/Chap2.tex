The interactions between solute atoms and crystalline defects such as dislocations, and grain boundaries play an essential role in determining physical, chemical and mechanical properties of solid-solution alloys. In recent years, the ability to predict solute segregation at high symmetry grain boundaries from first principles have been widely studied. However, previous algorithms have mainly focused on the simple grain boundary structures for dilute solute cases due to the costly computation power needed by density functional theory (DFT). Here, we present a general atomistic approach to optimize the structures and simulate solute segregation trends of grain boundaries in multiple component systems by the combination of a highly efficient genetic algorithm and the grand canonical ensemble, in which components are not restricted to dilute or stoichiometric cases. Different chemical potential can be used as input for creating different reservoirs for the grain boundary phases. In our study, thousands-atom grain boundary systems will be investigated by well-established empirical potentials (MEAM or EAM potentials) for Mg-based alloy systems, like Mg-Y or Mg-Zn, which are potential candidates for lightweight structural components as a result of their low density and high specific strength. Because of the complicated potential energy landscapes (PEL) coming from both geometric and occupational freedom, we will either average a good amount of small configurations by Boltzmann statistics based on their energy distributions for patterned segregation systems or use a large supercell to study cluster segregation systems across the grain boundaries. Final structures will then be used to investigate the effect of solute on mechanical behaviors of grain boundary systems.

\section{Introduction}
Solute atoms, whether they are added voluntarily for specific needs, inevitably remained as impurities after the synthesis, or introduced during the materials service, can affect various properties of alloys by changing the stability and mobility of crystalline defects.