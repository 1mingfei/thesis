\section{Conclusion}
\label{Chap:Al/Vac:section:Conc}


In this chapter, Ag-Mg-Zn ternary alloy is used as a model system to simulate the solute clustering kinetics in Al 7000 series alloys. We first demonstrate that the \acf{BEP} relationship, which suggests a simple linear relation between the activation barrier and the reaction energy of one elementary reaction step, fails to provide quantitatively accurate migration barriers of vacancies in these multi-component Al alloys. Then we develop a \ac{NN} model to predict vacancy migration barriers using the training data set of thousands of \ac{DFT} calculated barriers for different alloy configurations. 

A \ac{kMC} method based on this \ac{NN} model is used to study the early transition behavior from a supersaturated solid solution to solute clusters and \acf{GP} zones in Al-Mg-Zn alloys. A local super-basin method  Section \ref{Chap:Al/Vac:sec:LSKMC}, together with \ac{LRU} cache in Section \ref{Chap:Al/Vac:sec:LRU}, is also implemented to accelerate \ac{kMC} simulations. 

We also propose a pseudo-atoms approach to efficiently search the alloying strategy to slow down the solute clustering and the corresponding natural aging effects in Al 7000 series alloys. In this approach, a small number of pseudo-atoms with artificially designed ability to change vacancy migration barriers are added into the Al matrix, and the \ac{kMC} simulations are performed to check their effects to clustering kinetics (so-called ``sensitivity test''). 

\textcolor{red}{We also develop quantitative analysis methods to describe the chemical and structural properties of clusters. Using these quantitative analysis methods, we find that, 1) during the early stage of clustering, the first key step is to achieve Zn-rich clusters and these clusters become stable after 3 seconds in terms of size and chemistry; 2) for early-stage clusters, Zn/Mg ratio is converged to 1.2, which meets agreement with experimental value; 3) one possible reason for the behavior of slightly Zn-rich of early-stage clusters is vacancy migration barriers will reach a local minimum at intermediate Zn local concentration range.} 

At last, we propose a machine learning strategy based on the structural and chemical information of clusters and precipitates from \ac{kMC} simulations to predict the cluster strengthening and natural aging effects in future studies. \textcolor{red}{Using the bond counting model, we find the large variance of bond-wise energy could be possible for clustering strengthening. If by adding pseudo-atoms, more bonds with lower bond-wise energies can be achieved, then the strengthening effects can be slowed down.}