\section{Introduction}
\label{Chap:Al/Vac:section:Intro}

7000 series Al alloys developed for aerospace applications have high specific strength in the peak-aged condition. Their widespread implementation in the automotive industry for body and closure applications can achieve vehicle lightweighting goals if the challenges of the component forming and fabrication can be overcome. 

Conventional automotive mechanical forming processes such as stamping and hemming, involve room temperature forming of Al sheet alloys in the as-quenched state when alloys have high ductility and formability, followed by artificial aging during the paint bake operation. Due to fast (often within 30 minutes after quenching) and continuously varying natural aging characteristics of current 7000 series alloys, the formability decreases very rapidly and keeps changing as the aging time increases, necessitating costly steps such as warm stamping and coupled solutioning-quenching-stamping operations in a narrow time window.

